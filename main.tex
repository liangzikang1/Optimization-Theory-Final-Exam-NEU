% !TEX program = xelatex
\documentclass[12pt,a4paper,oneside]{ctexrep}

%——— 页面与常用宏包 ———
\usepackage[margin=2.2cm,headsep=10pt]{geometry}
\usepackage{amsmath,amssymb}
\usepackage{enumitem}
\usepackage{hyperref}
\usepackage{bookmark}
\usepackage{fancyhdr}
\usepackage{lastpage}
\usepackage{graphicx}
\usepackage{fontspec}
\usepackage{titling}
\usepackage{booktabs}

%——— 字体设置(macOS,宋体为 Songti SC)———
% 若为windows,请先搜索宋体英文名是什么?simsun?
\setCJKmainfont{Songti SC}
\setmainfont{Times New Roman}

%——— 章节(套卷)样式:将“章”显示为“第N套试卷” ——
\ctexset{
  chapter = {
    name   = {},
    number = \arabic{chapter},
    format = \LARGE\bfseries,
    beforeskip=20pt, afterskip=12pt,
  },
  contentsname = {目录}
}

%——— 目录仅显示“套卷”(chapter),不显示题目 ——
\setcounter{tocdepth}{0}

%——— 页眉页脚 ——
\pagestyle{fancy}
\fancyhf{}
\fancyhead[L]{使用前请对照原卷,如有出错邮箱联系}
\fancyhead[R]{\leftmark}
\fancyfoot[C]{\thepage}
\fancypagestyle{plain}{
  \fancyhf{}
  \fancyhead[L]{使用前请对照原卷,如有出错邮箱联系}
  \fancyhead[R]{\leftmark}
  \fancyfoot[C]{\thepage}
}

%——— 段落风格 ——
\setlength{\parindent}{2em}
\setlength{\parskip}{0.35em}

%——— 大题环境,每页两道题:第一题顶部,第二题中部 ——
\newcounter{problem}[chapter]
\renewcommand{\theproblem}{\arabic{problem}}

% 问题环境定义
\newenvironment{problem}[1][]{%
  \refstepcounter{problem}%
  % 第一题靠上,第二题居中
  \ifodd\value{problem}% 第一题
    \vspace*{1em}% 顶部稍微留白
  \else% 第二题
    \vspace*{0.2\textheight}% 页面中部位置
  \fi
  \par\noindent\textbf{\ \theproblem\if\relax\detokenize{#1}\relax\else\ (#1)\fi}\par\smallskip
}{%
  \par\medskip
  % 每两题后强制换页
  \ifodd\value{problem}\relax\else\clearpage\fi
}

%——— “套卷”环境 ——
\newenvironment{examset}[1][]{%
  \chapter{#1}%
  \setcounter{problem}{0}%
}{% 结束
}

\hypersetup{
  colorlinks=true,
  linkcolor=black,
  urlcolor=black,
  citecolor=black,
  pdftitle={数学试卷汇编}
}

%——————————— 示例 ———————————
\begin{document}

\title{数学试卷汇编}
\author{}
\date{\today}
\maketitle

\tableofcontents
\clearpage

\begin{examset}[2008-2009]
  \begin{problem}[程序设计]
一、本题仅为自己编写、运行并交了优化程序同学的必做题。答题情况将作为给程序成绩的依据。未交程序或未答此题的同学,程序成绩为零。

用你在已交优化程序中所用的编程语言,给下面的问题编写一个小程序。

\textbf{问题:} 设函数 $f(x,y)=x^{3}-y^{2}$,向量
$\mathbf{u}=(u_{1},u_{2},\ldots,u_{n})^{\mathrm T}$,$\mathbf{v}=(v_{1},v_{2},\ldots,v_{n})^{\mathrm T}$。
对每个分量,计算
\[
d_i=f(u_i,v_i)=u_i^{3}-v_i^{2},
\]
并将所有 $d_i>0$ 的值求和:
\[
S=\sum_{i=1}^{n}\mathbf{1}_{\{d_i>0\}}\,d_i
=\sum_{i=1}^{n}\mathbf{1}_{\{u_i^{3}-v_i^{2}>0\}}\,(u_i^{3}-v_i^{2}),
\]
其中 $\mathbf{1}_{\{\cdot\}}$ 为示性函数(条件成立取 $1$,否则取 $0$)。

\textbf{程序要求:} 必须有输入、输出结果数据语句,用循环语句编写计算语句。
\end{problem}


  \begin{problem}[试用两阶段单纯形法求解如下线性规划]
\[
\begin{aligned}
&\min\ -3x_1-2x_2\\
\text{s.t.}\quad
&3x_1+x_2=3,\\
&6x_1+3x_2\ge 7,\\
&x_1+2x_2\le 3,\\
&x_i\ge 0,\ i=1,2.
\end{aligned}
\]

  \end{problem}

  \begin{problem}
 有 $A,B$ 两种产品都需要经过前、后两道化学反应过程。每一个单位产品 $A$ 需要前道过程 $2$ 小时和后道过程 $3$ 小时;每一个单位产品 $B$ 需要前道过程 $3$ 小时和后道过程 $4$ 小时。可利用的前道过程时间是 $16$ 小时,后道过程时间是 $24$ 小时。每生产一个单位产品 $B$ 的同时,会生产两个单位的副产品 $C$,且不需要任何费用。副产品 $C$ 的一部分可以作为废料处理,其余的可以销售。出售单位产品 $A$ 可以获利 $4$ 元,出售单位产品 $B$ 可以获利 $10$ 元;出售单位副产品 $C$ 可以获利 $3$ 元,销毁单位副产品 $C$ 的费用是 $2$ 元。最多可售出 $5$ 个单位的副产品 $C$。问产品 $A,B$ 的产量、副产品 $C$ 的销售量和副产品 $C$ 的销毁量是多少,使利润达到最大?建立该问题的线性规划模型。
  \end{problem}


  \begin{problem}
    求函数 $f(x_1, x_2) = (x_1-2)^2 + (x_1-2x_2)^2$ 在点 $(1, 1)^T$ 处的 $\text{Taylor}$ 展开式 (写到三项).
  \end{problem}

  \begin{problem}
    对于极小化问题
$$ \min f(x_1, x_2) = 4x_1^2 + x_2^2 - x_1^2x_2 $$
判断 $\bar{x}_1 = [0, 0]^T$ 和 $\bar{x}_2 = [2\sqrt{2}, 4]^T$ 是否是该问题的局部极小点.
  \end{problem}

  \begin{problem}
对于线性规划问题

\[
\begin{aligned}
&\max\ -x_1+x_2\\
\text{s.t.}\quad
&2x_1+x_2-x_3=0,\\
&x_1-x_2+2x_3+2x_4=6,\\
&4x_2+x_3-x_4=4,\\
&x_i\ge 0,\ i=1,2,3,4.
\end{aligned}
\]

设 $B=(\bar a_1,\bar a_3,\bar a_4)$,$B$ 是否是基?如果 $B$ 是基,那么求出关于 $B$ 的基本解,并判断它是否是基本容许解。

  \end{problem}

  \begin{problem}
    下面是三个二元正定二次函数的等值线图. 从指定初始点 $\mathbf{x}_0$ 出发, 各图按指定算法, 画出求极小点的迭代路径. 其中最速下降法须迭代三次.
    \begin{figure}[htbp]
  \centering
  \includegraphics[width=1\linewidth]{fig/2008-1.png}
  \caption{不会p图见谅}
  \label{fig:example}
    \end{figure}
  \end{problem}
\clearpage
  \begin{problem}
    已知无约束优化问题
$$ \min f(x_1, x_2) = \frac{1}{2}x_1^2 + x_1x_2 + \frac{3}{2}x_2^2 + x_1 - 2x_2 + 1 $$
取初始点为 $\bar{x}_0 = [-4, 3]^T$, 用 DFP 算法迭代两次, 并判断最后一点是否为最优解.

\bigskip

公式:
$$ H_{k+1} = H_k + \frac{\bar{s}_k \bar{s}_k^T}{\bar{s}_k^T \bar{y}_k} - \frac{H_k \bar{y}_k \bar{y}_k^T H_k}{\bar{y}_k^T H_k \bar{y}_k}, \quad H_0 = \begin{bmatrix} 1 & 0 \\ 0 & 1 \end{bmatrix} $$
  \end{problem}

  \begin{problem}
    采用图上作业方式, 用步长加速法求无约束问题 $\min f(\bar{x})$. 初始点为图中标号 $1$ 的点. 步长取为图中单位格的长和宽. 极小点在 $x^*$ 处. 迭代到开始缩小步长为止. $f(\bar{x})$ 的等值线图如下所示.
\bigskip

要求: 1. 按下面的标识方式标出搜索过程中的全部基点、参考点和探测经厉点.
\begin{itemize}
    \item 基点 / 临参考点 / 基点与参考点重合点 / 探测经厉点
\end{itemize}
2. 按进行的先后顺序对基点、参考点进行序号 $1, 2, \cdots$.

    图片省略(太模糊)
  \end{problem}
\end{examset}

\begin{examset}[2009-2010]
  \begin{problem}
  试用两阶段单纯形法求解线性规划问题:
\[
\begin{array}{ll}
\min & Z = -2x_1 + 2x_3 \\
\text{s.t.} & x_1 - 2x_2 + x_3 \leq 2 \\
& -2x_1 + x_2 + x_3 \geq 3 \\
& 2x_1 - x_3 = 1 \\
& x_1, x_2, x_3 \geq 0
\end{array}
\]
  \end{problem}

  \begin{problem}
  用 Newton 法求解极小化问题:
$$ \min f(x_1, x_2) = 3x_1^3 + x_2^3 - x_1^2x_2 $$
\text{初始点取为 } $\bar{x}_0 = [1, 1]^T$ \text{, 迭代一次, 并说明 } $\bar{x}_1$ \text{ 是否是局部最优解吗?}
  \end{problem}

  \begin{problem}
    \text{用 F-R 共轭梯度法求解无约束优化问题:}
$$ \min f(x_1, x_2) = e^{x_1^2 + x_1x_2 + 2x_2^2} $$
\text{初始点取为 } $\bar{x}_0 = [2, -1]^T$ \text{, 已知第一次沿 } $\bar{p}_0 = -\nabla f(\bar{x}_0)$ \text{ 方向迭代得到的迭代点 } $\bar{x}_1 = [1, -1]^T$ \text{ 不是最优解. 求第二个迭代点 } ($\bar{x}_2$) \text{.}
  \end{problem}

  \begin{problem}
    \text{试用 DFP 法求解无约束优化问题}
$$ \min f(\bar{x}) = x_1^2 + 2x_2^2 - 2x_1x_2 $$
\text{初始点取为 } $\bar{x}_0 = [1, 1]^T$ \text{, 已知第一次沿 } $\bar{p}_0 = -\nabla f(\bar{x}_0)$ \text{ 方向迭代得到迭代点 } $\bar{x}_1 = [1, \pm 1]^T$ \text{.}
$$ (\text{公式:} H_{k+1} = H_k + \frac{\bar{s}_k \bar{s}_k^T}{\bar{s}_k^T \bar{y}_k} - \frac{H_k \bar{y}_k \bar{y}_k^T H_k}{\bar{y}_k^T H_k \bar{y}_k} \text{, } H_0 = \begin{bmatrix} 1 & 0 \\ 0 & 1 \end{bmatrix}.) $$
  \end{problem}

  \begin{problem}
    \text{考虑约束最优化问题}
$$ \min f(x_1, x_2) = x_1^2 - 2x_1x_2 + 4x_2^2 $$
\text{s.t.}\begin{align*}
3x_1 + x_2 &\geq 2 \\
2x_1 - 5x_2 &\geq -10 \\
x_1 - 2x_2 &\leq 2 \\
x_1, x_2 &\geq 0
\end{align*}
\text{初始点取为 } $\bar{x}_0 = [0, 2]^T$ \text{. 用 Zoutendijk 法求许可方向迭代一次.}
  \end{problem}

  \begin{problem}
    \text{用外部罚函数法求解约束问题}
$$ \min f(x_1, x_2) = (x_1 - x_2)^2 - 5x_1 - x_2 $$
\text{s.t.}
\begin{align*}
2 - x_1 - x_2 &\geq 0 \\
x_2 &\geq 0
\end{align*}
  \end{problem}

  \begin{problem}
    \text{用最小二乘法解方程组:}
$$ \begin{cases} x_1 + x_2 = 1 \\ -2x_1 + x_2 = 0 \\ x_1 + 2x_2 = 4 \end{cases} $$
  \end{problem}

  \begin{problem}
    \text{试判断向量 } $\bar{p} = [1, 0, 1]^T$ \text{ 是否是如下线性约束}
\begin{align*}
x_1 + 2x_2 - x_3 &= 6 \\
-2x_1 + 5x_2 &\geq 4 \\
x_1 + x_3 &\leq 2 \\
x_i &\geq 0, i=1, 2, 3
\end{align*}
\text{在点 } $\bar{x} = [2, 2, 0]^T$ \text{ 处的容许方向向量.}
  \end{problem}

  \begin{problem}
    \text{利用 K-T 条件求解约束问题}
$$ \min f(x_1, x_2) = e^{-x_1} + e^{-x_2} $$
\text{s.t.}
$$ x_1 + x_2 \leq 1 $$
  \end{problem}
\end{examset}

\begin{examset}[2010-2011]
    \begin{problem}
        \text{用两阶段单纯形法解如下线性规划}
$$ \min 3x_1 - x_2 $$
\text{s.t.}
\begin{align*}
x_1 + x_2 + x_3 &\leq 6 \\
-2x_1 + x_2 - x_3 &\geq 1 \\
3x_2 + x_3 &= 9 \\
x_i &\geq 0, i=1, 2, 3
\end{align*}
\text{并回答: 该线性规划目标函数的等值——是————}
    \end{problem}

    \begin{problem}
        \text{用 Newton 法解无约束问题}
$$ \min f(x_1, x_2) = (x_1 - 1)^4 + (x_1^2 - x_2)^2 $$
\text{初始点取为 } $\bar{x}_0 = [1, 2]^T$ \text{, 迭代一次求 } $\bar{x}_1$ \text{, 并说明 } $\bar{x}_1$ \text{ 是否为最优解.}
    \end{problem}

    \begin{problem}
        \text{三、(15 分) 用 F-R 共轭梯度法解无约束问题}
$$ \min f(\bar{x}) = 3x_1^2 + 2x_1x_2 + x_2^2 - 6x_1 + 2x_2 $$
\text{初始点取为 } $\bar{x}_0 = [1, 0]^T$ \text{.}
\newline
\text{回答: F-R 共轭梯度法是一种共轭方向方法, 具有————}\text{ 步终止性. 本题的目标函数为————}, 使用共轭梯度法至多迭代———— \text{ 次即可求到最优解. 本题的目标函数的等值——是—————}
    \end{problem}

    \begin{problem}
        \text{用步长加速法求解无约束极小化问题}
$$ \min f(x_1, x_2) = 4x_1^2 + x_2^2 - x_1x_2 $$
\text{初始点取为 } $\bar{x}_0 = [-1, 1]^T$ \text{. 初始步长向量 } $\bar{s}_0 = [1, 1]^T$ \text{. 迭代到开始缩小步长为止.}
\newline
\text{根据计算过程回答: 步长加速法主要包括 加速 和 修正 两个基本过程. 如果有 } $f(\bar{b}_1) < f(\bar{b}_0)$ \text{, 则说明 } $\bar{b}_1 - \bar{b}_0$ \text{ 是 } $\bar{b}_0$ \text{ 处的 下降方向.}
    \end{problem}
    
    \begin{problem}
        已测得变量 $t$ 与 $y$ 的 5 组数据如下:
\begin{center}
\begin{tabular}{|c|c|c|c|c|c|}
    \hline
    $t$ & 0 & 1 & 2 & 3 & 4 \\
    \hline
    $y$ & 1 & 3 & 5 & 4 & 2 \\
    \hline
\end{tabular}
\end{center}

\begin{enumerate}
    \item 根椐这 5 组数据,试说说用一次函数,还是二次函数作 $t$ 与 $y$ 之间的拟合更好?
    \item 用在 (1) 中你所选择的拟合函数,建立 $t$ 与 $y$ 之间的最小二乘模型。(注:不求值)
\end{enumerate}
    \end{problem}

    \begin{problem}
        对于方程组
\[
\left\{
\begin{aligned}
t_1 + x_2 &= 4, \\
-t_1 + 2x_2 &= -1, \\
3t_1 - x_2 &= 1,
\end{aligned}
\right.
\]
求其最小二乘解,并给出结果判断该方程组是否含有解。
    \end{problem}

    \begin{problem}
        求如下约束问题

\begin{align*}
    \min_{x_1, x_2} \quad & f(x_1, x_2) = \ln x_1 - x_2 \\
    \text{s.t.} \quad & x_1 - x_2^2 \ge 1 \\
\end{align*}

的 K-T 点,并利用凸规划的 K-T 点定理,判断求到的 K-T 点是否为问题的最优解。
    \end{problem}

    \begin{problem}
        用 Zoutendijk 容许方向法解如下约束问题
\begin{align*}
    \min_{x_1, x_2} \quad & x_1^2 + 2x_2^2 - x_1 x_2 \\
    \text{s.t.} \quad & x_1 + x_2 = 2, \\
    \quad & x_1 + 4x_2 \le 5, \\
    \quad & x_1, x_2 \ge 0.
\end{align*}
初始点取为 $\bar{\mathbf{x}} = [1, 1]^T$,迭代一步求出 $\bar{\mathbf{x}}_1$。
    \end{problem}

    \begin{problem}
        用乘子法解如下约束问题
\begin{align*}
    \min_{x_1, x_2} \quad & f(x_1, x_2) = (x_1 - x_2)^2 - 5x_1 - x_2 \\
    \text{s.t.} \quad & x_1 + x_2 \le 2, \\
    \quad & x_1, x_2 \ge 0. \quad (\text{隐含约束})
\end{align*}
并回答:乘子法中的罚因子与外部罚函数法中的罚因子的本质区别。


\textbf{公式:} $F(\bar{\mathbf{x}}, \bar{\boldsymbol{\lambda}}, \bar{\boldsymbol{\nu}}, \mu) = f(\bar{\mathbf{x}}) - \sum_{j=1}^{l} \lambda_j h_j(\bar{\mathbf{x}}) + \mu \sum_{j=1}^{l} h_j^2(\bar{\mathbf{x}}) + \frac{1}{4\mu} \sum_{i=1}^{m} \{[\max(0, \bar{\nu}_i - 2\mu g_i(\bar{\mathbf{x}}))]^2 - \bar{\nu}_i^2\}$.

\begin{align*}
    \lambda_j^{(t+1)} &= \lambda_j^{(t)} - 2\mu h_j(\bar{\mathbf{x}}_t), \quad j=1, 2, \dots, l; \\
    \nu_i^{(t+1)} &= \max(0, \nu_i^{(t)} - 2\mu g_i(\bar{\mathbf{x}}_t)), \quad i=1, 2, \dots, m.
\end{align*}
    \end{problem}
\end{examset}


\begin{examset}[2011-2012]
    \begin{problem}
        试用两阶段单纯形法解线性规划
\begin{align*}
    \min \quad & 2x_1 + 3x_2 + 4x_3 \\
    \text{s.t.} \quad & x_1 + 2x_2 + x_3 \ge 3 \\
    \quad & 2x_1 - x_2 + 3x_3 \ge 4 \\
    \quad & x_i \ge 0, \quad i=1, 2, 3.
\end{align*}
    \end{problem}

    \begin{problem}
        设一个求极小的线性规划的容许集 $D$、目标函数的负梯度向量 $-\bar{c}$ 及初始点 $\bar{x}_0$ 的位置如右图所示。试在图上画出至少 3 条等值线,及从初始点 $\bar{x}_0$ 到最优点的迭代路径(要求标出各迭代点)。

        图省略了(请看原卷)
    \end{problem}

    \begin{problem}
        下面是三个二元正定二次函数的等值线图。从指定初始点 $\mathbf{x}_0$ 出发,各图按指定算法画出求极小点的迭代路径。其中最速下降法要求只迭代二次。
        \begin{figure}[htbp]
  \centering
  \includegraphics[width=1\linewidth]{fig/2011-1}
    \end{figure}

    \end{problem}

    \begin{problem}
        用 F-R 共轭梯度法解无约束极小化问题
\[
\min f(\bar{\mathbf{x}}) = -\frac{1}{2} x_1^2 + \frac{1}{2} x_2^2 + x_3^2 - x_1 x_2,
\]
初始点取为 $\bar{\mathbf{x}}_0 = [1, 1, 1]^T$。按 F-R 共轭梯度法现已迭代出 $\bar{\mathbf{x}}_1 = [\frac{1}{5}, 1, \frac{1}{5}]^T$,要求继续迭代一次,求出 $\bar{\mathbf{x}}_2$,并判断 $\bar{\mathbf{x}}_2$ 是否为局部最优解。
    \end{problem}

    \begin{problem}
        用 DFP 法解无约束极小化问题
\[
\min f(\bar{\mathbf{x}}) = x_1^2 + 2x_2^2 - x_1^2 x_2
\]
初始点取为 $\bar{\mathbf{x}}_0 = [1, 1]^T$。按 DFP 法现已迭代出 $\bar{\mathbf{x}}_1 = [1, 1/4]^T$,要求继续迭代一次,求出 $\bar{\mathbf{x}}_2$,并判断 $\bar{\mathbf{x}}_2$ 是否为局部最优解。

\vspace{1em}
\noindent
\textbf{(公式:}
\begin{align*}
    H_{k+1} &= H_k + \frac{\mathbf{s}_k \mathbf{s}_k^T}{\mathbf{s}_k^T \mathbf{y}_k} - \frac{H_k \mathbf{y}_k \mathbf{y}_k^T H_k^T}{\mathbf{y}_k^T H_k \mathbf{y}_k}, \\
    H_0 &= \begin{bmatrix} 1 & 0 \\ 0 & 1 \end{bmatrix}.
\end{align*}
\textbf{注:} $\sqrt{146} \approx 12.08$,$\sqrt{148} \approx 12.17$)
    \end{problem}

    \begin{problem}
        用加速法解无约束极小化问题
\[
\min f(\bar{\mathbf{x}}) = 4x_1^2 + x_2^2.
\]
设初始点 $\bar{\mathbf{x}}_0 = [1, 2.5]^T$,初始步长方向 $\bar{\mathbf{s}}_0 = [1, 1]^T$。要求迭代到模式移动结束为止。
    \end{problem}

    \begin{problem}
        试述容许方向法的 3 个主要过程是:
\begin{enumerate}
    \item 确定当前迭代点处的\underline{\hspace{3cm}}。
    \item \underline{\hspace{3cm}},得到下一个迭代点。
    \item 判断新\underline{\hspace{3cm}}是否为 K-T 点。
\end{enumerate}

    求解如下线性约束问题
\begin{align*}
    \min \quad & f(x_1, x_2, x_3) = x_1^3 + 2x_2^2 - x_3^2 - 3x_2 x_3 \\
    \text{s.t.} \quad & x_1 + x_2 + x_3 = 1, \\
    \quad & 3x_1 + 4x_2 + 2x_3 \le 6, \\
    \quad & x_i \ge 0, \quad i=1, 2, 3.
\end{align*}

设初始点取为 $\bar{\mathbf{x}}_0 = [1, 0, 0]^T$。

\begin{enumerate}
    \item 检验 $\bar{\mathbf{p}}_0 = [-2, 1, 1]^T$ 是否为 $\bar{\mathbf{x}}_0$ 处的下降容许方向。
    \item 如果 $\bar{\mathbf{p}}_0 = [-2, 1, 1]^T$ 是 $\bar{\mathbf{x}}_0$ 处的下降容许方向,那么沿 $\bar{\mathbf{p}}_0$ 作直线搜索,求出下一迭代点 $\bar{\mathbf{x}}_1$。然后停止计算;否则,直接停止计算。
\end{enumerate}
    \end{problem}

    \begin{problem}
        简述外部罚函数法的“罚”思想。鉴于乘子法是外部罚函数法的改进方法,简述“改进”的本质。

        试用乘子法解如下约束问题
\begin{align*}
    \min \quad & f(x_1, x_2) = x_1^2 + x_1 x_2 + x_2^2 \\
    \text{s.t.} \quad & x_1 + x_2 \ge 1.
\end{align*}

\vspace{1em}
\noindent
\textbf{(公式:}
\begin{itemize}
    \item 扩展拉格朗日函数 $F(\bar{\mathbf{x}}, \bar{\boldsymbol{\lambda}}, \bar{\boldsymbol{\nu}}, \mu) = f(\bar{\mathbf{x}}) - \sum_{j=1}^{l} \lambda_j h_j(\bar{\mathbf{x}}) + \mu \sum_{j=1}^{l} h_j^2(\bar{\mathbf{x}}) + \frac{1}{4\mu} \sum_{i=1}^{m} \{[\max(0, \bar{\nu}_i - 2\mu g_i(\bar{\mathbf{x}}))]^2 - \bar{\nu}_i^2\}$.
    \item 等式乘子更新: $\lambda_j^{(k+1)} = \max(0, \dots - 2\mu h_j(\bar{\mathbf{x}}_k)), \quad j=1, 2, \dots, l.$
    \item 不等式乘子更新: $\nu_i^{(k+1)} = \max(0, \nu_i^{(k)} - 2\mu g_i(\bar{\mathbf{x}}_k)), \quad i=1, 2, \dots, m.$
\end{itemize}
    \end{problem}

    \begin{problem}
        试利用 K-T 条件解如下约束极小化问题
\begin{align*}
    \min \quad & f(x_1, x_2, x_3) = x_1 + x_2 - x_3 \\
    \text{s.t.} \quad & x_1 - x_2 = 0, \\
    \quad & 1 - x_1^2 - x_2^2 - x_3^2 \ge 0.
\end{align*}
    \end{problem}

    \begin{problem}
        某工厂向用户提供发动机,合同规定交货量和交货日期分别是:第一季度末交 40 台,第二季度末交 60 台,第三季度末交 80 台。工厂每季的最大生产能力为 100 台。每季的生产费用函数为 $f(x) = 50x + 0.2x^2$ (元), $x$ 为生产台数。如果某季生产过剩,则多余的发动机可用于下季交货,但工厂需要支付存储费用:每台每季 4 元。试建立工厂能够完成合同而费用最少的数学模型(假定第一季度开始时发动机无存货)。
    \end{problem}
\end{examset}

\begin{examset}[2013-2014]
    \begin{problem}
        用两阶段单纯形法解线性规划
\begin{align*}
    \min \quad & 5x_1 + 3x_2 + 2x_3 \\
    \text{s.t.} \quad & 3x_1 + 2x_2 + x_3 \ge 2 \\
    \quad & x_1 - x_2 + 2x_3 \ge 3 \\
    \quad & x_i \ge 0, \quad i=1, 2, 3.
\end{align*}
    \end{problem}

    \begin{problem}
        采集到关于变量 $x, y, z$ 的一组数据:
\begin{center}
\begin{tabular}{|c|c|c|c|c|c|c|}
    \hline
    $x$ & 3 & 5 & 6 & 8 & 12 & 14 \\
    \hline
    $y$ & 16 & 10 & 7 & 4 & 3 & 2 \\
    \hline
    $z$ & 90 & 72 & 54 & 42 & 30 & 12 \\
    \hline
\end{tabular}
\end{center}

试建立 $z$ 关于 $x, y$ 的线性回归最小二乘模型。
    \end{problem}

    \begin{problem}
        判断 $\mathbf{p} = [1, 1, -2]^T$ 是线性约束
\begin{align*}
    x_1 + x_2 + x_3 &= 4, \\
    -3x_1 + 10x_2 + 6x_3 &\le 10, \\
    6x_1 - x_2 + 4x_3 &\ge 15
\end{align*}
在点 $\bar{\mathbf{x}}_0 = [2, 1, 1]^T$ 处的容许方向向量吗?
    \end{problem}

    \begin{problem}
        (1) 利用最优性条件,求如下无约束问题
\[
\min f(x_1, x_2) = x_1^2 + x_2^2 - x_1^2 x_2
\]
的严格局部极小点。

    (2)用 F-R 法解无约束极小化问题
\[
\min f(x_1, x_2) = x_1^2 + x_2^2 - x_1^2 x_2.
\]
设初始点为 $\bar{\mathbf{x}}_0 = [1, 1]^T$,且按 F-R 法已求出 $\bar{\mathbf{x}}_1 = [1, 1/2]^T$。试再迭代一次,求出 $\bar{\mathbf{x}}_2$。(不要把分数与根式化成小数)
    \end{problem}

    \begin{problem}
        用 DFP 法解无约束极小化问题
\[
\min f(\mathbf{x}) = (x_1 - 2)^2 + 2x_2^2.
\]
设初始点为 $\bar{\mathbf{x}}_0 = [0, 1]^T$,且按 DFP 法已得到 $\bar{\mathbf{x}}_1 = [4/3, -4/3]^T$。

\vspace{1em}
\noindent
\textbf{(校正公式:} $H_0 = \text{单位矩阵}$,
\[
H_{k+1} = H_k + \frac{\mathbf{s}_k \mathbf{s}_k^T}{\mathbf{s}_k^T \mathbf{y}_k} - \frac{H_k \mathbf{y}_k \mathbf{y}_k^T H_k^T}{\mathbf{y}_k^T H_k \mathbf{y}_k}
\]
(不要把分数与根式化成小数)
    \end{problem}

    \begin{problem}
        采用图上作业方式,用步长加速法解无约束问题 $\min f(\bar{\mathbf{x}})$。$f(\bar{\mathbf{x}})$ 的等值线图如下图所示,标号 $1$ 处为初始点。设步长向量的分量均取 $1$ (图中网格为单位格),迭代到开始缩小步长为止。

\textbf{注意:} 
\begin{enumerate}
    \item[(a)] 在下图左下方所示的直角坐标系上标出探测搜索方向的先后顺序;
    \item[(b)] 按指定标识方式: 
        \begin{itemize}
            \item[$\bullet$] 基点
            \item[$\square$] 参考点
            \item[$\blacksquare$] 基点与参考点的重合点
            \item[$\circ$] 探测经历点
        \end{itemize}
        标出作业过程中的全部基点、参考点和探测经历点;
    \item[(c)] 按进行的顺序,仅对基点标号 $1, 2, \dots$。
\end{enumerate}
\begin{figure}[htbp]
  \centering
  \includegraphics[width=1\linewidth]{fig/2013-1}
    \end{figure}
    \end{problem}

    \begin{problem}
        (1) 求极小化约束问题
\[
\min f(x_1, x_2) = -\ln(1+x_1) - 2\ln(1+x_2)
\]
\[
\text{s.t. } x_1 + x_2 \le 2.
\]
的 K-T 点。

\textbf{(2)} 某公司有 3 个建筑工地要开工,每个工地位置用平面坐标 $(a, b)$ 表示,距离单位:公里;及水泥用量 $d$ (单位:吨) 由下表给出:
\begin{center}
\begin{tabular}{|c|c|c|c|}
    \hline
    \textbf{工地} & \textbf{1} & \textbf{2} & \textbf{3} \\
    \hline
    $(a, b)$ & $(2, 2)$ & $(8, 1)$ & $(1, 4)$ \\
    \hline
    $d$ (吨) & 3 & 5 & 4 \\
    \hline
\end{tabular}
\end{center}
现拟建造一个料场,但储量备量为 6 吨。假设料场到工地之间均有直线道路相连。问如何选址和调配调用量,能使总的吨公里数最少(仅建数学模型)?
    \end{problem}

    \begin{problem}
        求解约束问题
\begin{align*}
    \min \quad & f(\bar{\mathbf{x}}) = x_1^2 + x_2^2 - 2x_1 - 4x_2; \\
    \text{s.t.} \quad & 2x_1 - x_2 \le 1, \\
    \quad & x_1 + x_2 \le 2, \\
    \quad & x_1 \ge 0, x_2 \ge 0.
\end{align*}
设初始点取为 $\bar{\mathbf{x}}_0 = [1, 1]^T$,试用 Zoutendijk 容许方向法迭代一次,求出下一迭代点 $\bar{\mathbf{x}}_1$。
    \end{problem}

    \begin{problem}
        (1) 外部罚函数法的惩罚方式是针对\rule{3cm}{0.4pt} 点惩罚, 而对\rule{3cm}{0.4pt}点不惩罚, 罚因子的特点\rule{10cm}{0.4pt}

(2) 用乘子法解约束问题 问题时有如下:

\[
\begin{aligned}
\min & \quad f(x_1, x_2) = \frac{1}{2} x_1^2 + \frac{1}{2} x_2^2 + x_1; \\
\text{s.t.} & \quad x_1 + x_2 \geq 1, \\
& \quad x_1 x_2 = 70.
\end{aligned}
\]

(公式: $F(\bar{x}, \bar{\lambda}, \bar{v}, \mu) = f(\bar{x}) - \sum_{j=1}^{2} \lambda_j h_j(\bar{x}) + \mu \sum_{j=1}^{2} h_j^2(\bar{x}) + \frac{1}{4\mu} \sum_{i=1}^{m} \left\{[\max(0, v_i^k - 2\mu s_i(\bar{x}_k))]^2 - v_i^{k^2}\right\}$

\[
\bar{\lambda}_j^{(k+1)} = \bar{\lambda}_j^{(k)} - 2\mu h_j(\bar{x}_k), j=1,2,\dots,l; \quad v_i^{(k+1)} = \max(0, v_i^{(k)} - 2\mu s_i(\bar{x}_k)), i=1,2,\dots,m.
\]
    \end{problem}
\end{examset}

\begin{examset}[2014-2015]
    \begin{problem}
        用两阶段单纯形法解如下线性规划

\[
\begin{aligned}
\min \quad & -3x_1 + x_2 + x_3; \\
\text{s.t.} \quad & -4x_1 + x_2 + 2x_3 \geq 3, \\
& x_1 + 2x_2 + x_3 \leq 11, \\
& x_1 + x_3 = 1, \\
& -2x_1 + x_3 = 1, \\
& x_i \geq 0, \quad i=1,2,3.
\end{aligned}
\]
    \end{problem}

    \begin{problem}
        用图解法求解非线性规划问题

\[
\begin{aligned}
\min \quad & x_1^2 + x_2^2 - 2x_1 + 1; \\
\text{s.t.} \quad & x_1 - x_2^2 \leq 1, \\
& x_1 - x_2 \geq 2, \\
& x_1, x_2 \geq 0.
\end{aligned}
\]
    \end{problem}

    \begin{problem}
        已测得变量 $i$ 与 $y$ 的 5 组数据:

\begin{table}[h]
\centering
\begin{tabular}{c|ccccc}
\toprule
$i$   & $-1$ & $-0.7$ & $0$ & $0.7$ & $1$ \\
$y$   & $-0.9$ & $-1$ & $-2$ & $-3.5$ & $-4.4$ \\
\bottomrule
\end{tabular}
\end{table}

(1) 根据这 5 组数据,试从以下 4 个函数中选择最合适的一个函数拟合:

\[
y = x_1(i + x_2), \quad y = x_1 i^2 + x_2 i + x_3, \quad y = x_1 e^{x_2 i}, \quad y = \ln(x_1 + x_2 i)
\]

(2) 用你在 (1) 中选择的拟合函数,建立 $y$ 与 $i$ 之间的最小二乘模型。

(3) 判断你在 (2) 中建立的是否是线性最小二乘模型.如果是,写出 $A, b$.
    \end{problem}

    \begin{problem}
        用 F-R 共轭梯度法求解无约束问题

\[
\min x_1^2 + x_2^2 - x_1^2 x_2 .
\]

初始点为 $\bar{x}_0 = [1, 1]^T$,现已求出 $\bar{x}_1 = [1, 1/2]^T$,再迭代一次,求 $\bar{x}_2$ 并讨论第 2 个迭代点是否为最优解。
    \end{problem}

    \begin{problem}
        用乘子法求解约束问题

\[
\begin{aligned}
\min \quad & f(x) = x_1^2 + (x_2 - 1)^2; \\
\text{s.t.} \quad & x_1 \geq 20 \quad \Rightarrow \quad x_1 - 20 \geq 0, \\
& x_2 \geq 2.
\end{aligned}
\]

公式: $F(\bar{x}, \bar{\lambda}, \bar{v}, \mu) = f(\bar{x}) - \sum_{j=1}^{l} \lambda_j h_j(\bar{x}) + \mu \sum_{j=1}^{l} h_j^2(\bar{x}) + \frac{1}{4\mu} \sum_{i=1}^{m} \left\{[\max(0, \bar{v}_i - 2\mu s_i(\bar{x}))]^2 - \bar{v}_i^2\right\}$

\[
\bar{\lambda}_j^{(k+1)} = \bar{\lambda}_j^{(k)} - 2\mu h_j(\bar{x}_k), \ j=1,2,\dots,l;
\]

\[
\bar{v}_i^{(k+1)} = \max(0, \bar{v}_i^{(k)} - 2\mu s_i(\bar{x}_k)), \ i=1,2,\dots,m.
\]
    \end{problem}

    \begin{problem}
        用 DFP 法求解无约束问题

\[
\min x_1^2 + x_2^2 + \frac{1}{2} x_3^2 + x_1 x_2 - x_3 .
\]

初始点为 $\bar{x}_0 = [-2,1]^T$,现已求出 $\bar{x}_1 = [-1/2,1]^T$,再迭代一次,求 $\bar{x}_2$,并解答 2 个问题:(1) $\bar{x}_2$ 是最优解吗,为什么?(2) 如果 $\bar{x}_2$ 不是最优解,那么求到最优解还需要多少次迭代?

公式:
\[
\bar{H}_{k+1} = \bar{H}_k + \frac{\bar{s}_k \bar{s}_k^T}{\bar{s}_k^T \bar{y}_k} - \frac{\bar{H}_k \bar{y}_k \bar{y}_k^T \bar{H}_k}{\bar{y}_k^T \bar{H}_k \bar{y}_k}, \quad
\bar{H}_0 = \begin{bmatrix} 1 & 0 \\ 0 & 1 \end{bmatrix} .
\]
    \end{problem}


    \begin{problem}
        设函数 $f(\bar{x}) = \frac{1}{2} \bar{x}^T Q \bar{x} + \bar{b}^T \bar{x} + c$,其中 $Q$ 是正定矩阵,$\bar{x}_*$ 为满足 $\nabla f(\bar{x}_*) = \bar{0}$ 的任意一点,从 $\bar{x}_0$ 出发,沿方向 $\bar{p} = -Q^{-1} \nabla f(\bar{x}_0)$ 对 $f(\bar{x})$ 作直线搜索,求最优步长因子 $t_0$(从直线搜索到的极小点 $\bar{z}_1$,试问 $\bar{z}_1$ 与 $f(\bar{x})$ 是什么关系?
    \end{problem}

    \begin{problem}
        采用图上作业方式,用步长加速法求解无约束问题 $\min f(\bar{x})$.$f(\bar{x})$ 的等值线图如下图所示,标号 1 处为初始点,设初始步长向量的分量均为 1(图中网格为单位格),迭代到开始缩小步长为止.要求:

(a) 在下方所示直角坐标系上标出探测的先后顺序;

(b) 按下面指定的标识方式,标出作业过程中的全部基点、参考点和探测经历点:

\begin{itemize}[label=]
\item $\square$ 参考点 \quad $\bullet$ 基点 \quad $\circ$ 探测经历点 \quad $\blacksquare$ 基点与参考点重合的点
\end{itemize}

(c) 按进行的顺序,仅对焦点标号:1, 2, \dots

\begin{figure}[htbp]
  \centering
  \includegraphics[width=1\linewidth]{fig/2014-1}
    \end{figure}
    \end{problem}

    \begin{problem}
        用 z-容许方向法求解约束问题

\[
\begin{aligned}
\min \quad & x_1^2 - x_1 x_2 + x_2^2 - 2x_1; \\
\text{s.t.} \quad & -3x_1 - x_2 \geq -3, \\
& x_1 \geq 0, x_2 \geq 0.
\end{aligned}
\]

初始点为 $\bar{x}_0 = [2/3, 0]^T$,迭代一次,求 $\bar{x}_1$.
    \end{problem}

    \begin{problem}
        考虑约束问题

\[
\begin{aligned}
\min \quad & x_1^2 - x_1 x_2 + x_2^2, \\
\text{s.t.} \quad & x_1 - 1 \geq 0, \\
& x_1 + x_2 - 6 = 0.
\end{aligned}
\]

利用下述条件求 K-T 点,并判断所求出的 K-T 点是否是全局最优解。
    \end{problem}

    \begin{problem}
        考虑线性规划

\[
\begin{aligned}
\min \quad & -4x_1 - x_2 + x_3 - x_4; \\
\text{s.t.} \quad & x_1 - 3x_3 + x_4 = 4, \\
& 2x_1 + x_2 + x_3 = 10, \\
& x_i \geq 0, \quad i=1,\dots,4.
\end{aligned}
\]

(1) 求该问题的所有极点并验证解;

(2) 已知该问题有最优解,试问:“可否通过 1 的方式求得最优解?为什么?”

(3) 如果能由 1 的方式求出最优解,那么求出最优解.
    \end{problem}
\end{examset}

\begin{examset}[2015-2016]
    \begin{problem}
        用两阶段单纯形法解下面的线性规划

\[
\begin{aligned}
\min \quad & -5x_1 + x_2 + 2x_3; \\
\text{s.t.} \quad & 3x_1 - x_2 + 2x_3 \leq 3, \\
& -x_1 + 2x_2 + 3x_3 \geq 4, \\
& x_1 + x_2 + x_3 = 2, \\
& -x_1 + x_2 + x_3 \leq 1, \\
& x_1, x_2, x_3 \geq 0.
\end{aligned}
\]
    \end{problem}

    \begin{problem}
        (1) 图(a), (b)显示的是两个最小化线性规划的容许集、初始点和目标函数的梯度方向,试画出单纯形法的迭代路径(并说明每种情况),图(c)是一个极小化问题的等值线图;  
试判断点 $\bar{x}_0$ 方向,画出最速下降法的前 3 次迭代路径。
        \begin{figure}[htbp]
  \centering
  \includegraphics[width=1\linewidth]{fig/2015-1}
    \end{figure}
    \end{problem}

    \begin{problem}
        利用最小二乘法解线性方程组

\[
\begin{cases}
x_1 + x_2 + x_3 = 4, \\
2x_1 - 3x_2 - x_3 = -2, \\
-x_1 + 2x_2 - 2x_3 = -4, \\
-x_1 + x_3 = 1.
\end{cases}
\]
    \end{problem}

    \begin{problem}
        利用极小点的判定条件求下面优化问题的极小点.
\[
\min x_1^3 + 3x_1 x_2^2 - 15x_1 - 12x_2
\]
    \end{problem}

    \begin{problem}
        用 DFP 法求解优化问题

\[
\min x_1^2 + 2x_2^2 + 2x_1 x_2 - x_1 + x_2 .
\]

初始点为 $\bar{x}_0 = [0, 0]^T$.

公式:
\[
\bar{H}_{k+1} = \bar{H}_k + \frac{\bar{s}_k \bar{s}_k^T}{\bar{s}_k^T \bar{y}_k} - \frac{\bar{H}_k \bar{y}_k \bar{y}_k^T \bar{H}_k}{\bar{y}_k^T \bar{H}_k \bar{y}_k}, \quad
\bar{H}_0 = \begin{bmatrix} 1 & 0 \\ 0 & 1 \end{bmatrix} .
\]
    \end{problem}

    \begin{problem}
        采用图上作业,用步长加速法求解无约束问题 $\min f(\bar{x})$.下图是 $f(\bar{x})$ 的等值线图,标号 1 处为初始点,设初始步长向量的分量均为 1(图中网格为单位格),迭代到开始缩小步长为止.要求:

(a) 在下方所示直角坐标系上标出探测方向的顺序;

(b) 按下面指定的标识方式,标出作业过程中的全部基点、参考点和探测经历点,对基点按进行的先后依次标号 1, 2, \dots

\begin{itemize}[label=]
\item $\square$ 参考点 \quad $\bullet$ 基点 \quad $\circ$ 探测经历点 \quad $\blacksquare$ 基点与参考点重合的点
\end{itemize}

(c) 以“$\rightarrow$”标出模式移动。

    \begin{figure}[htbp]
  \centering
  \includegraphics[width=1\linewidth]{fig/2015-2}
    \end{figure}
    \end{problem}

    \begin{problem}
        (1) 用 Z-容许方向法求解约束问题

\[
\begin{aligned}
\min \quad & x_1^2 + 4x_2^2 - x_2; \\
\text{s.t.} \quad & 2x_1 + 3x_2 \leq 5, \\
& x_1 + x_2 = 2, \\
& x_1 \geq 0, x_2 \geq 0.
\end{aligned}
\]

设初始点为 $\bar{x}_0 = [1,1]^T$,迭代一次求 $\bar{x}_1$.
    \end{problem}

    \begin{problem}
        考虑问题

\[
\min x_1^3 + 3x_1 x_2^2 - 15x_1 - 12x_2 .
\]

设初始点为 $\bar{x}_0 = (2,3)^T$.用 Newton 法迭代一次.
    \end{problem}

    \begin{problem}
        用 F-R 共轭梯度法求解优化问题

\[
\min 2(x_1 - x_2)^2 + (x_2 - 1)^2 .
\]

设初始点为 $\bar{x}_0 = [0,0]^T$,迭代二次求出 $\bar{x}_2$(提示:最优步长因子 $t_0$ 是 2 的负整数次幂).
    \end{problem}

    \begin{problem}
        设 $\bar{x} \in \mathbb{R}^n, \bar{c} \neq \bar{0}$,考虑问题

\[
\begin{array}{ll}
\min & \bar{c}^T \bar{x}, \\
\text{s.t.} & \bar{x}^T \bar{x} \geq 70 \text{ s.t. } \bar{x}^T \bar{x} \leq 1.
\end{array}
\]

求该问题的 K-T 点,并判断其是否为全局极小点。
    \end{problem}

    \begin{problem}
        用外部罚函数法求解约束问题

\[
\begin{aligned}
\min \quad & -\ln(x_1 + 1) - \ln(x_2 + 1); \\
\text{s.t.} \quad & x_1 + x_2 \leq 2, \\
& x_i \geq 0.
\end{aligned}
\]
    \end{problem}

    \begin{problem}
        考虑问题.

\[
\begin{aligned}
\min \quad & x_1; \\
\text{s.t.} \quad & 16 - (x_1 - 4)^3 - x_2^2 \geq 0, \\
& (x_1 - 3)^2 + (x_2 - 2)^3 - 13 \geq 0.
\end{aligned}
\]

问:$\bar{x}_1 = (0,0)$ 与 $\bar{x}_2 = (3,2-\sqrt{3})$ 是 K-T 点吗?为什么?如果是 K-T 点,则是否为全局极小点?
    \end{problem}
\end{examset}

\begin{examset}[2016-2017]
    \begin{problem}
        用两阶段单纯形法解线性规划

\[
\begin{array}{ll}
\min & z = -4x_1 - x_2 \\
& \\
\text{s.t.} & 
\left\{
\begin{array}{l}
x_1 + 2x_2 \leq 4 \\
-4x_1 - 3x_2 \leq -6 \\
3x_2 + x_3 \geq 3 \\
x_1, x_2 \geq 0
\end{array}
\right.
\end{array}
\]
    \end{problem}

    \begin{problem}
        1. 简述迭代格式 $x_{k+1} = \overline{x}_k + t_k \overline{p}_k$ ($k=0,1,\ldots$) 对于 $x$ 受许方向法与最速下降法的区别,以及 $t_k$ 取了什么值才能保证 $\overline{x}_k$ 是下一次迭代的允许点。

        \textbf{2.} 试图表示的克极小化问题 $\min f(x)$,$R$ 表示容许域。(1) 在图中画出点 $\overline{x}_1$,$\overline{x}_2$ 的所有起作用的束函数的梯度方向;(2) $\overline{x}_1$,$\overline{x}_2$ 是否 $K-T$ 点?
        \begin{figure}[htbp]
  \centering
  \includegraphics[width=0.45\linewidth]{fig/2016-1}
    \end{figure}

    \end{problem}
        
    \begin{problem}
        \textbf{(1)} 用图解法求解下面问题;\textbf{(2)} 将它化为标准线性规划问题。

\[
\begin{array}{ll}
\max z = x_1 + 2x_2 \\
& \\
\text{s.t.}
\begin{cases}
2x_1 + 3x_2 \leq 6 \\
x_1 + x_2 \geq 1 \\
x_2 \geq 0
\end{cases}
\end{array}
\]
    \end{problem}

    \begin{problem}
        用 Newton 法求解极小化问题

\[
\min x_1^2 x_2 + 2x_2 x_3^4 - 4x_1 x_2,
\]

初始点取为 $x_0 = [1, 1]^T$,迭代一次得到迭代点 $\overline{x}_1$,并说明 $\overline{x}_1$ 是局部最优解吗?
    \end{problem}

    \begin{problem}
        设初始点为 $\overline{x}_0 = [2, -1]^T$。用 F-R 共轭梯度法求解无约束极小化问题

\[
\min x_1^2 + 2x_1 x_2 + 2x_2^2.
\]
    \end{problem}

    \begin{problem}
        试用 DFP 法求解无约束优化问题

\[
\min x_1^3 + 2x_2^3 - 2x_1 x_2 + x_2^5,
\]

初始点取为 $\overline{x}_0 = [1, 1]^T$。已知:一次迭代 $p_0 = -\nabla f(\overline{x}_0)$ 方向迭代得到迭代点 $\overline{x}_1 = \begin{bmatrix} \frac{1}{3} & \frac{1}{3} \end{bmatrix}^T$ 否是最优解。求第二个迭代点 $\overline{x}_2$,并判断 $\overline{x}_2$ 是否局部最优解。

(校正公式:$H_0 = I$ 单位矩阵,$H_{k+1} = H_k + \frac{s_k s_k^T}{s_k^T y_k} - \frac{H_k y_k y_k^T H_k}{y_k^T H_k y_k}$)
    \end{problem}

    \begin{problem}
        用步长加速法求解无约束极小化问题

\[
\min (x_1 - 1)^4 + 2x_2^2
\]

初始点取为 $\overline{x}_0 = [4, 3]^T$,初始步长向量 $\overline{s}_0 = [1, 1]^T$,迭代到开始缩小步长为止
    \end{problem}

    \begin{problem}
        1. 利用最优性条件,求解约束问题

\[
\begin{array}{ll}
\min & -x_1 + x_2 \\
\text{s.t.} & x_1 + x_2 \leq 1, \\
& x_1 + x_2 = 1.
\end{array}
\]
    \end{problem}

    \begin{problem}
        下表给出了某厂节能煤耗技术改造后生产甲产品过程中记录的产量 $x$(吨)与相应的生产能耗 $y$(吨标准煤)的几组对照数据

\[
\begin{array}{c|cccc}
x & 3 & 4 & 5 & 6 \\
\hline
y & 2.5 & 3 & 4 & 4.5
\end{array}
\]

(1) 请根据上表提供的数据,用最小二乘法求出 $y$ 与 $x$ 的线性回归方程;  

(2) 已知该厂技改前 100 吨甲产品的生产能耗为 90 吨标准煤,试根据 (1) 求出的线性回归方程,预测生产 100 吨甲产品的生产能耗比技改前降低多少吨标准煤。
    \end{problem}


    \begin{problem}
        求解约束问题

\[
\begin{array}{ll}
\min & 2x_1^2 + x_2 x_3 + 2x_2^2 \\
\text{s.t.} & 2x_1 + 7x_2 \leq 10, \\
& x_1 - 3x_2 \leq 4, \\
& x_1 + 2x_2 \geq 4, \\
& x_i \geq 0.
\end{array}
\]

(1) 设初始点取为 $\overline{x}_0 = [4, 0]^T$,试用 $z$ 容许方向法迭代一次,求出下一迭代点 $\overline{x}_1$;  

(2) $z$ 容许方向法的迭代终止准则是什么?  

(3) 并预测 $\overline{x}_0$ 和 $\overline{x}_1$ 是否满足终止准则。
    \end{problem}

    \begin{problem}
        用外部罚函数法求解约束问题

\[
\begin{array}{ll}
\min & f(x) = (x_1 - 2)^2 + (x_2 - 1)^2 \\
\text{s.t.} & x_1 + 2x_2 \geq 7 \\
& 2x_1 - x_2 \geq 4
\end{array}
\]
    \end{problem}
\end{examset}
\end{document}


% \begin{figure}[htbp]
%   \centering
%   \includegraphics[width=1\linewidth]{fig/2011-1}
%     \end{figure}
