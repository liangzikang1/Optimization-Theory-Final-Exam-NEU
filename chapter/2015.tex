\begin{examset}[2015-2016]
    \begin{problem}
        用两阶段单纯形法解下面的线性规划

\[
\begin{aligned}
\min \quad & -5x_1 + x_2 + 2x_3; \\
\text{s.t.} \quad & 3x_1 - x_2 + 2x_3 \leq 3, \\
& -x_1 + 2x_2 + 3x_3 \geq 4, \\
& x_1 + x_2 + x_3 = 2, \\
& -x_1 + x_2 + x_3 \leq 1, \\
& x_1, x_2, x_3 \geq 0.
\end{aligned}
\]
    \end{problem}

    \begin{problem}
        (1) 图(a), (b)显示的是两个最小化线性规划的容许集、初始点和目标函数的梯度方向,试画出单纯形法的迭代路径(并说明每种情况),图(c)是一个极小化问题的等值线图;  
试判断点 $\bar{x}_0$ 方向,画出最速下降法的前 3 次迭代路径。
        \begin{figure}[htbp]
  \centering
  \includegraphics[width=1\linewidth]{fig/2015-1}
    \end{figure}
    \end{problem}

    \begin{problem}
        利用最小二乘法解线性方程组

\[
\begin{cases}
x_1 + x_2 + x_3 = 4, \\
2x_1 - 3x_2 - x_3 = -2, \\
-x_1 + 2x_2 - 2x_3 = -4, \\
-x_1 + x_3 = 1.
\end{cases}
\]
    \end{problem}

    \begin{problem}
        利用极小点的判定条件求下面优化问题的极小点.
\[
\min x_1^3 + 3x_1 x_2^2 - 15x_1 - 12x_2
\]
    \end{problem}

    \begin{problem}
        用 DFP 法求解优化问题

\[
\min x_1^2 + 2x_2^2 + 2x_1 x_2 - x_1 + x_2 .
\]

初始点为 $\bar{x}_0 = [0, 0]^T$.

公式:
\[
\bar{H}_{k+1} = \bar{H}_k + \frac{\bar{s}_k \bar{s}_k^T}{\bar{s}_k^T \bar{y}_k} - \frac{\bar{H}_k \bar{y}_k \bar{y}_k^T \bar{H}_k}{\bar{y}_k^T \bar{H}_k \bar{y}_k}, \quad
\bar{H}_0 = \begin{bmatrix} 1 & 0 \\ 0 & 1 \end{bmatrix} .
\]
    \end{problem}

    \begin{problem}
        采用图上作业,用步长加速法求解无约束问题 $\min f(\bar{x})$.下图是 $f(\bar{x})$ 的等值线图,标号 1 处为初始点,设初始步长向量的分量均为 1(图中网格为单位格),迭代到开始缩小步长为止.要求:

(a) 在下方所示直角坐标系上标出探测方向的顺序;

(b) 按下面指定的标识方式,标出作业过程中的全部基点、参考点和探测经历点,对基点按进行的先后依次标号 1, 2, \dots

\begin{itemize}[label=]
\item $\square$ 参考点 \quad $\bullet$ 基点 \quad $\circ$ 探测经历点 \quad $\blacksquare$ 基点与参考点重合的点
\end{itemize}

(c) 以“$\rightarrow$”标出模式移动。

    \begin{figure}[htbp]
  \centering
  \includegraphics[width=1\linewidth]{fig/2015-2}
    \end{figure}
    \end{problem}

    \begin{problem}
        (1) 用 Z-容许方向法求解约束问题

\[
\begin{aligned}
\min \quad & x_1^2 + 4x_2^2 - x_2; \\
\text{s.t.} \quad & 2x_1 + 3x_2 \leq 5, \\
& x_1 + x_2 = 2, \\
& x_1 \geq 0, x_2 \geq 0.
\end{aligned}
\]

设初始点为 $\bar{x}_0 = [1,1]^T$,迭代一次求 $\bar{x}_1$.
    \end{problem}

    \begin{problem}
        考虑问题

\[
\min x_1^3 + 3x_1 x_2^2 - 15x_1 - 12x_2 .
\]

设初始点为 $\bar{x}_0 = (2,3)^T$.用 Newton 法迭代一次.
    \end{problem}

    \begin{problem}
        用 F-R 共轭梯度法求解优化问题

\[
\min 2(x_1 - x_2)^2 + (x_2 - 1)^2 .
\]

设初始点为 $\bar{x}_0 = [0,0]^T$,迭代二次求出 $\bar{x}_2$(提示:最优步长因子 $t_0$ 是 2 的负整数次幂).
    \end{problem}

    \begin{problem}
        设 $\bar{x} \in \mathbb{R}^n, \bar{c} \neq \bar{0}$,考虑问题

\[
\begin{array}{ll}
\min & \bar{c}^T \bar{x}, \\
\text{s.t.} & \bar{x}^T \bar{x} \geq 70 \text{ s.t. } \bar{x}^T \bar{x} \leq 1.
\end{array}
\]

求该问题的 K-T 点,并判断其是否为全局极小点。
    \end{problem}

    \begin{problem}
        用外部罚函数法求解约束问题

\[
\begin{aligned}
\min \quad & -\ln(x_1 + 1) - \ln(x_2 + 1); \\
\text{s.t.} \quad & x_1 + x_2 \leq 2, \\
& x_i \geq 0.
\end{aligned}
\]
    \end{problem}

    \begin{problem}
        考虑问题.

\[
\begin{aligned}
\min \quad & x_1; \\
\text{s.t.} \quad & 16 - (x_1 - 4)^3 - x_2^2 \geq 0, \\
& (x_1 - 3)^2 + (x_2 - 2)^3 - 13 \geq 0.
\end{aligned}
\]

问:$\bar{x}_1 = (0,0)$ 与 $\bar{x}_2 = (3,2-\sqrt{3})$ 是 K-T 点吗?为什么?如果是 K-T 点,则是否为全局极小点?
    \end{problem}
\end{examset}