\begin{examset}[2016-2017]
    \begin{problem}
        用两阶段单纯形法解线性规划

\[
\begin{array}{ll}
\min & z = -4x_1 - x_2 \\
& \\
\text{s.t.} & 
\left\{
\begin{array}{l}
x_1 + 2x_2 \leq 4 \\
-4x_1 - 3x_2 \leq -6 \\
3x_2 + x_3 \geq 3 \\
x_1, x_2 \geq 0
\end{array}
\right.
\end{array}
\]
    \end{problem}

    \begin{problem}
        1. 简述迭代格式 $x_{k+1} = \overline{x}_k + t_k \overline{p}_k$ ($k=0,1,\ldots$) 对于 $x$ 受许方向法与最速下降法的区别,以及 $t_k$ 取了什么值才能保证 $\overline{x}_k$ 是下一次迭代的允许点。

        \textbf{2.} 试图表示的克极小化问题 $\min f(x)$,$R$ 表示容许域。(1) 在图中画出点 $\overline{x}_1$,$\overline{x}_2$ 的所有起作用的束函数的梯度方向;(2) $\overline{x}_1$,$\overline{x}_2$ 是否 $K-T$ 点?
        \begin{figure}[htbp]
  \centering
  \includegraphics[width=0.45\linewidth]{fig/2016-1}
    \end{figure}

    \end{problem}
        
    \begin{problem}
        \textbf{(1)} 用图解法求解下面问题;\textbf{(2)} 将它化为标准线性规划问题。

\[
\begin{array}{ll}
\max z = x_1 + 2x_2 \\
& \\
\text{s.t.}
\begin{cases}
2x_1 + 3x_2 \leq 6 \\
x_1 + x_2 \geq 1 \\
x_2 \geq 0
\end{cases}
\end{array}
\]
    \end{problem}

    \begin{problem}
        用 Newton 法求解极小化问题

\[
\min x_1^2 x_2 + 2x_2 x_3^4 - 4x_1 x_2,
\]

初始点取为 $x_0 = [1, 1]^T$,迭代一次得到迭代点 $\overline{x}_1$,并说明 $\overline{x}_1$ 是局部最优解吗?
    \end{problem}

    \begin{problem}
        设初始点为 $\overline{x}_0 = [2, -1]^T$。用 F-R 共轭梯度法求解无约束极小化问题

\[
\min x_1^2 + 2x_1 x_2 + 2x_2^2.
\]
    \end{problem}

    \begin{problem}
        试用 DFP 法求解无约束优化问题

\[
\min x_1^3 + 2x_2^3 - 2x_1 x_2 + x_2^5,
\]

初始点取为 $\overline{x}_0 = [1, 1]^T$。已知:一次迭代 $p_0 = -\nabla f(\overline{x}_0)$ 方向迭代得到迭代点 $\overline{x}_1 = \begin{bmatrix} \frac{1}{3} & \frac{1}{3} \end{bmatrix}^T$ 否是最优解。求第二个迭代点 $\overline{x}_2$,并判断 $\overline{x}_2$ 是否局部最优解。

(校正公式:$H_0 = I$ 单位矩阵,$H_{k+1} = H_k + \frac{s_k s_k^T}{s_k^T y_k} - \frac{H_k y_k y_k^T H_k}{y_k^T H_k y_k}$)
    \end{problem}

    \begin{problem}
        用步长加速法求解无约束极小化问题

\[
\min (x_1 - 1)^4 + 2x_2^2
\]

初始点取为 $\overline{x}_0 = [4, 3]^T$,初始步长向量 $\overline{s}_0 = [1, 1]^T$,迭代到开始缩小步长为止
    \end{problem}

    \begin{problem}
        1. 利用最优性条件,求解约束问题

\[
\begin{array}{ll}
\min & -x_1 + x_2 \\
\text{s.t.} & x_1 + x_2 \leq 1, \\
& x_1 + x_2 = 1.
\end{array}
\]
    \end{problem}

    \begin{problem}
        下表给出了某厂节能煤耗技术改造后生产甲产品过程中记录的产量 $x$(吨)与相应的生产能耗 $y$(吨标准煤)的几组对照数据

\[
\begin{array}{c|cccc}
x & 3 & 4 & 5 & 6 \\
\hline
y & 2.5 & 3 & 4 & 4.5
\end{array}
\]

(1) 请根据上表提供的数据,用最小二乘法求出 $y$ 与 $x$ 的线性回归方程;  

(2) 已知该厂技改前 100 吨甲产品的生产能耗为 90 吨标准煤,试根据 (1) 求出的线性回归方程,预测生产 100 吨甲产品的生产能耗比技改前降低多少吨标准煤。
    \end{problem}


    \begin{problem}
        求解约束问题

\[
\begin{array}{ll}
\min & 2x_1^2 + x_2 x_3 + 2x_2^2 \\
\text{s.t.} & 2x_1 + 7x_2 \leq 10, \\
& x_1 - 3x_2 \leq 4, \\
& x_1 + 2x_2 \geq 4, \\
& x_i \geq 0.
\end{array}
\]

(1) 设初始点取为 $\overline{x}_0 = [4, 0]^T$,试用 $z$ 容许方向法迭代一次,求出下一迭代点 $\overline{x}_1$;  

(2) $z$ 容许方向法的迭代终止准则是什么?  

(3) 并预测 $\overline{x}_0$ 和 $\overline{x}_1$ 是否满足终止准则。
    \end{problem}

    \begin{problem}
        用外部罚函数法求解约束问题

\[
\begin{array}{ll}
\min & f(x) = (x_1 - 2)^2 + (x_2 - 1)^2 \\
\text{s.t.} & x_1 + 2x_2 \geq 7 \\
& 2x_1 - x_2 \geq 4
\end{array}
\]
    \end{problem}
\end{examset}