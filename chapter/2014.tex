\begin{examset}[2014-2015]
    \begin{problem}
        用两阶段单纯形法解如下线性规划

\[
\begin{aligned}
\min \quad & -3x_1 + x_2 + x_3; \\
\text{s.t.} \quad & -4x_1 + x_2 + 2x_3 \geq 3, \\
& x_1 + 2x_2 + x_3 \leq 11, \\
& x_1 + x_3 = 1, \\
& -2x_1 + x_3 = 1, \\
& x_i \geq 0, \quad i=1,2,3.
\end{aligned}
\]
    \end{problem}

    \begin{problem}
        用图解法求解非线性规划问题

\[
\begin{aligned}
\min \quad & x_1^2 + x_2^2 - 2x_1 + 1; \\
\text{s.t.} \quad & x_1 - x_2^2 \leq 1, \\
& x_1 - x_2 \geq 2, \\
& x_1, x_2 \geq 0.
\end{aligned}
\]
    \end{problem}

    \begin{problem}
        已测得变量 $i$ 与 $y$ 的 5 组数据:

\begin{table}[h]
\centering
\begin{tabular}{c|ccccc}
\toprule
$i$   & $-1$ & $-0.7$ & $0$ & $0.7$ & $1$ \\
$y$   & $-0.9$ & $-1$ & $-2$ & $-3.5$ & $-4.4$ \\
\bottomrule
\end{tabular}
\end{table}

(1) 根据这 5 组数据,试从以下 4 个函数中选择最合适的一个函数拟合:

\[
y = x_1(i + x_2), \quad y = x_1 i^2 + x_2 i + x_3, \quad y = x_1 e^{x_2 i}, \quad y = \ln(x_1 + x_2 i)
\]

(2) 用你在 (1) 中选择的拟合函数,建立 $y$ 与 $i$ 之间的最小二乘模型。

(3) 判断你在 (2) 中建立的是否是线性最小二乘模型.如果是,写出 $A, b$.
    \end{problem}

    \begin{problem}
        用 F-R 共轭梯度法求解无约束问题

\[
\min x_1^2 + x_2^2 - x_1^2 x_2 .
\]

初始点为 $\bar{x}_0 = [1, 1]^T$,现已求出 $\bar{x}_1 = [1, 1/2]^T$,再迭代一次,求 $\bar{x}_2$ 并讨论第 2 个迭代点是否为最优解。
    \end{problem}

    \begin{problem}
        用乘子法求解约束问题

\[
\begin{aligned}
\min \quad & f(x) = x_1^2 + (x_2 - 1)^2; \\
\text{s.t.} \quad & x_1 \geq 20 \quad \Rightarrow \quad x_1 - 20 \geq 0, \\
& x_2 \geq 2.
\end{aligned}
\]

公式: $F(\bar{x}, \bar{\lambda}, \bar{v}, \mu) = f(\bar{x}) - \sum_{j=1}^{l} \lambda_j h_j(\bar{x}) + \mu \sum_{j=1}^{l} h_j^2(\bar{x}) + \frac{1}{4\mu} \sum_{i=1}^{m} \left\{[\max(0, \bar{v}_i - 2\mu s_i(\bar{x}))]^2 - \bar{v}_i^2\right\}$

\[
\bar{\lambda}_j^{(k+1)} = \bar{\lambda}_j^{(k)} - 2\mu h_j(\bar{x}_k), \ j=1,2,\dots,l;
\]

\[
\bar{v}_i^{(k+1)} = \max(0, \bar{v}_i^{(k)} - 2\mu s_i(\bar{x}_k)), \ i=1,2,\dots,m.
\]
    \end{problem}

    \begin{problem}
        用 DFP 法求解无约束问题

\[
\min x_1^2 + x_2^2 + \frac{1}{2} x_3^2 + x_1 x_2 - x_3 .
\]

初始点为 $\bar{x}_0 = [-2,1]^T$,现已求出 $\bar{x}_1 = [-1/2,1]^T$,再迭代一次,求 $\bar{x}_2$,并解答 2 个问题:(1) $\bar{x}_2$ 是最优解吗,为什么?(2) 如果 $\bar{x}_2$ 不是最优解,那么求到最优解还需要多少次迭代?

公式:
\[
\bar{H}_{k+1} = \bar{H}_k + \frac{\bar{s}_k \bar{s}_k^T}{\bar{s}_k^T \bar{y}_k} - \frac{\bar{H}_k \bar{y}_k \bar{y}_k^T \bar{H}_k}{\bar{y}_k^T \bar{H}_k \bar{y}_k}, \quad
\bar{H}_0 = \begin{bmatrix} 1 & 0 \\ 0 & 1 \end{bmatrix} .
\]
    \end{problem}


    \begin{problem}
        设函数 $f(\bar{x}) = \frac{1}{2} \bar{x}^T Q \bar{x} + \bar{b}^T \bar{x} + c$,其中 $Q$ 是正定矩阵,$\bar{x}_*$ 为满足 $\nabla f(\bar{x}_*) = \bar{0}$ 的任意一点,从 $\bar{x}_0$ 出发,沿方向 $\bar{p} = -Q^{-1} \nabla f(\bar{x}_0)$ 对 $f(\bar{x})$ 作直线搜索,求最优步长因子 $t_0$(从直线搜索到的极小点 $\bar{z}_1$,试问 $\bar{z}_1$ 与 $f(\bar{x})$ 是什么关系?
    \end{problem}

    \begin{problem}
        采用图上作业方式,用步长加速法求解无约束问题 $\min f(\bar{x})$.$f(\bar{x})$ 的等值线图如下图所示,标号 1 处为初始点,设初始步长向量的分量均为 1(图中网格为单位格),迭代到开始缩小步长为止.要求:

(a) 在下方所示直角坐标系上标出探测的先后顺序;

(b) 按下面指定的标识方式,标出作业过程中的全部基点、参考点和探测经历点:

\begin{itemize}[label=]
\item $\square$ 参考点 \quad $\bullet$ 基点 \quad $\circ$ 探测经历点 \quad $\blacksquare$ 基点与参考点重合的点
\end{itemize}

(c) 按进行的顺序,仅对焦点标号:1, 2, \dots

\begin{figure}[htbp]
  \centering
  \includegraphics[width=1\linewidth]{fig/2014-1}
    \end{figure}
    \end{problem}

    \begin{problem}
        用 z-容许方向法求解约束问题

\[
\begin{aligned}
\min \quad & x_1^2 - x_1 x_2 + x_2^2 - 2x_1; \\
\text{s.t.} \quad & -3x_1 - x_2 \geq -3, \\
& x_1 \geq 0, x_2 \geq 0.
\end{aligned}
\]

初始点为 $\bar{x}_0 = [2/3, 0]^T$,迭代一次,求 $\bar{x}_1$.
    \end{problem}

    \begin{problem}
        考虑约束问题

\[
\begin{aligned}
\min \quad & x_1^2 - x_1 x_2 + x_2^2, \\
\text{s.t.} \quad & x_1 - 1 \geq 0, \\
& x_1 + x_2 - 6 = 0.
\end{aligned}
\]

利用下述条件求 K-T 点,并判断所求出的 K-T 点是否是全局最优解。
    \end{problem}

    \begin{problem}
        考虑线性规划

\[
\begin{aligned}
\min \quad & -4x_1 - x_2 + x_3 - x_4; \\
\text{s.t.} \quad & x_1 - 3x_3 + x_4 = 4, \\
& 2x_1 + x_2 + x_3 = 10, \\
& x_i \geq 0, \quad i=1,\dots,4.
\end{aligned}
\]

(1) 求该问题的所有极点并验证解;

(2) 已知该问题有最优解,试问:“可否通过 1 的方式求得最优解?为什么?”

(3) 如果能由 1 的方式求出最优解,那么求出最优解.
    \end{problem}
\end{examset}